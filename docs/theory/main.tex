\documentclass{article}
\usepackage[utf8]{inputenc}
\usepackage{times}
\usepackage[margin=1.75in]{geometry}


\usepackage{multicol}
\usepackage{graphicx}
\graphicspath{{academic/figs/}}

\usepackage{dblfloatfix}
% \usepackage{algorithm}
% \usepackage{algorithmic}
\usepackage[ruled]{algorithm2e}
\newcommand\mycommfont[1]{\footnotesize\sffamily{#1}}
\newcommand\myprocname[1]{\rmfamily{#1}}
\SetCommentSty{mycommfont}
\SetFuncSty{myprocname}

\usepackage{changepage}

\usepackage{mathtools}
\usepackage{mathrsfs}
\usepackage{amsmath, amsthm}
\usepackage{bm}
\usepackage{bbm}
\usepackage{dsfont}
\usepackage{amssymb}

% numbers option provides compact numerical references in the text.
\usepackage[numbers]{natbib}
% \usepackage[sectionbib,round]{natbib}
% \usepackage{chapterbib}

\usepackage{upgreek}
\usepackage{multirow}
\usepackage{amsfonts}
\usepackage{here}
\usepackage[normalem]{ulem}
% \usepackage{fontspec}
\usepackage{booktabs}
\usepackage{color}
\usepackage{subcaption}
\usepackage{minitoc}
\usepackage[font=small, skip=5pt]{caption}
\usepackage[shortlabels]{enumitem}
\usepackage{listings}
\lstset{
  mathescape,
  frame=tblr,
  aboveskip=3mm,
  belowskip=3mm,
  showstringspaces=false,
  columns=flexible,
  basicstyle={\ttfamily\small},
  numbers=none,
  numberstyle=\tiny\color{gray},
  keywordstyle=\color{blue},
  commentstyle=\color{dkgreen},
  stringstyle=\color{mauve},
  breaklines=true,
  breakatwhitespace=true,
  tabsize=3
}

\usepackage{xurl}
\usepackage{tikz}
\usepackage[bookmarks=true]{hyperref}
\usepackage{csquotes}
\usepackage{xcolor} % Hyperlink color
\hypersetup{
    colorlinks,
    linkcolor={red!50!black},
    citecolor={blue!80!black},
    urlcolor={blue!80!black}
}

\pdfinfo{
   /Author (Kaiyu Zheng)
   /Title  (One Big Collection)
}
\usepackage{cleveref}
\usepackage{pdfpages}





% \makeatletter
% \def\@IEEEsectpunct{.\ \,}
% \def\paragraph{\@startsection{paragraph}{4}{\z@}{1.5ex plus 1.5ex minus 0.5ex}%
% {0ex}{\normalfont\normalsize\bfseries}}
% \makeatother



% New commands
\DeclareMathOperator*{\argmax}{argmax}
\DeclareMathOperator*{\argmin}{argmin}
\newcommand{\etal}{\emph{et~al.}}
\newcommand{\E}{\mathbb{E}}
\newcommand{\roll}{\theta_{\text{roll}}}
\newcommand{\pitch}{\theta_{\text{pitch}}}
\newcommand{\yaw}{\theta_{\text{yaw}}}
\def\imp{\rightarrow}
\newcommand\XOR{\mathbin{\char`\^}}
\newcommand\IFF{\leftrightarrow}
\newcommand\Equiv{\Longleftrightarrow}
\newcommand\modt{\text{ mod }}
\newcommand{\modp}[1]{(\text{mod }$#1$)}
\newcommand{\MOD}[2]{\text{$#1$ (mod $#2$)}}
\newcommand{\IMP}{\Rightarrow}
\newcommand{\bksh}{\textbackslash}

\newcommand{\tr}{\textnormal{tr}}
\newcommand{\tl}{\tilde}
%% \newcommand{\Pr}{\text{Pr}}
\newcommand{\Var}{\text{Var}}
\newcommand{\atan}{\text{atan}}
\newcommand{\Alpha}{\text{A}}
\newcommand{\Norm}[1]{\lVert #1 \rVert}
\newcommand{\erf}{\text{erf}}
\renewcommand{\Re}{\text{Re }}
\renewcommand{\Im}{\text{Im }}
\newcommand{\Ln}{\text{Ln }}
\newcommand{\ul}[1]{\underline{#1}}

\renewcommand{\d}{\partial}
\newcommand{\D}{\Delta}
\newcommand{\e}{\text{e}}



\DeclarePairedDelimiter\ceil{\lceil}{\rceil}
\DeclarePairedDelimiter\floor{\lfloor}{\rfloor}


\newtheorem{problem}{Problem}
\newtheorem{definition}{Definition}
\newtheorem{theorem}{Theorem}
\newtheorem{corollary}{Corollary}
\newtheorem*{lemma}{Lemma}
\newcommand{\norm}[1]{\left\lVert#1\right\rVert}
\newcommand{\eg}{\textit{e.g.}}
\newcommand{\ie}{\textit{i.e.}}

\newcommand{\Val}{\textsc{Val}}
\newcommand{\Normz}{\textsc{Norm}}
\newcommand{\Children}{\textsc{Ch}}

\newcommand{\1}[1]{\mathds{1}\left[#1\right]}

\newcommand{\todo}[1]{{\footnotesize\color{red}[{\bf TODO: } \textsf{#1}]}}
% People's comments
\definecolor{purple}{rgb}{0.858, 0.08, 0.85}
\definecolor{darkgreen}{rgb}{0.08, 0.55, 0.08}
\definecolor{blue1}{rgb}{0.2, 0.2, 0.6}
\definecolor{green1}{rgb}{0.2, 0.6, 0.2}
\definecolor{green2}{rgb}{0.1, 0.4, 0.1}
\definecolor{dkgreen}{rgb}{0,0.6,0}
\definecolor{gray}{rgb}{0.5,0.5,0.5}
\definecolor{mauve}{rgb}{0.58,0,0.82}

\makeatletter
\newcommand*\bigcdot{\mathpalette\bigcdot@{.5}}
\newcommand*\bigcdot@[2]{\mathbin{\vcenter{\hbox{\scalebox{#2}{$\m@th#1\bullet$}}}}}
\makeatother

\newcommand{\td}[1]{\tilde{#1}}

\title{Theory}
\author{Kaiyu Zheng}
\date{\today}

\begin{document}
\maketitle

\section{Preliminaries}

The preliminary material is based on \citet{kaelbling1998planning} and \citet{pineau2003point}.

\subsection{POMDP}
A POMDP is defined as a tuple
$\langle S, A, \Omega, T, O, R, \gamma \rangle$: $S, A, \Omega$ are the state,
action, and observation spaces; $T(s,a,s')=\Pr(s'|s,a)$, $O(s',a,o)=\Pr(o|s',a)$
and $R(s,a)\in\mathbb{R}$ are the transition, observation, and reward functions;
$\gamma$ is the discount factor. The agent has partial observability of the
state and maintains a \emph{belief state}, denoted as $b$, which maps a state to
a probability. The belief state is a sufficient statistic for the history $h=(ao)_{1:t-1}$
%$h=(a_1,o_1,\cdots, a_{t_1},o_{t-1})$,
meaning that $\Pr(s|b,h)=\Pr(s|b)$\footnote{One can define $b(s)=\Pr(s|h)$ and regard $b$ as a vector with size $|S|$. So $\Pr(s|b)=\Pr(s|h)$.}.


The solution to a POMDP is a policy $\pi:B\rightarrow A$ that maps a belief
state to an action. Equivalently, a policy can be represented as a \emph{policy tree}\footnote{See illustration in Figure 4 of
  \citet{kaelbling1998planning}.}, where the root node contains the immediate
next action, and it has $|O|$ children, such that each child is a subtree that represents the action selection policy upon receiving an observation. The depth of the tree is the planning horizon.

\subsection{Value function of a POMDP}
It is clear to express the value function of a POMDP using the
policy tree representation\footnote{See section 4.1 of \citet{kaelbling1998planning}}.
Given a policy tree $p$, let $a(p)$ denote the action rooted at the tree, and
$o(p)$ denote the subtree along the branch for observation $o$. Then, the \emph{value}
of this policy, under the true world state $s$, is the expected discounted cumulative reward,
\begin{align}
V_p(s) = R(s,a(p)) + \gamma \sum_{s'\in S}T(s,a(p),s')\sum_{o\in\Omega}O(s',a(p),o) V_{o(p)}(s')
\end{align}
The value of the policy under a belief state $b$ is the expected value over states:
\begin{align}
V_p(b) = \E_{s\sim b} [V_p(s)] = \sum_{s\in S}b(s)V_p(s)
\end{align}
Define $\alpha_p=[ V_p(s_1)\ V_p(s_2)\ \cdots\ V_p(s_n)]$ where $s_1,\cdots,s_n\in S$, and regard the belief state $b$ a vector. We have
\begin{align}
V_p(b) = b\cdot \alpha_p
\end{align}
The vector $\alpha_p$ is commonly referred to as the ``$\alpha$-vector'' in the literature. Every policy $p$ has a corresponding $\alpha$-vector. Suppose $\mathcal{P}$ is the set of all possible policies with horizon $t$, then the optimal value at a belief $b$ under this planning horizon is
\begin{align}
V_t(b) = \max_{p\in\mathcal{P}} b\cdot \alpha_p
\end{align}
It is important to understand the geometric interpretation of this value function\footnote{See Figure 5 and 6 in \citet{kaelbling1998planning}.}. The first insight is that $V_p(b)$ is a hyperplane under $S$.\footnote{A hyperplane under a $d$ dimensional space is a subspace with dimension $d-1$.} That means each $\alpha$-vector corresponds to a hyperplane. The second insight is that the belief space can be divided into regions, where the belief points in each region have an $\alpha$-vector which leads to the maximum value. In other words, the set of $\alpha$-vectors where each dominates (i.e. leads to maximum value at) a non-empty belief region forms the \emph{optimal policy}. The value function $V_t:B\rightarrow \mathbb{R}$, as a result, is a piecewise linear and convex function.

An alternative interpretation leads to another, equivalent, form of $V_t(b)$ \cite{pineau2003point}. The value at a belief state $b$ can be thought of as the dot product between $b$ and the dominating $\alpha$-vector at this belief point. This dominating $\alpha$-vector should correspond to a policy tree optimal at $b$. This means selecting the best root action, and the best subtree for every child, which should lead to the highest value among all possible subtrees. Hence,
\begin{align}
\begin{split}
V_t(b) &= \max_{a\in A} \Bigg[ \sum_{s\in S} R(s,a)b(s)\\
&\qquad\qquad+ \gamma \sum_{o\in\Omega} \max_{\alpha'\in \Gamma_{t-1}} \sum_{s\in S} \sum_{s'\in S} T(s,a,s')O(s',a,o)\alpha'(s')b(s)  \Bigg]
\end{split}
\end{align}
where $\Gamma_{t-1}=\{\alpha_1,\alpha_2,\cdots,\alpha_n\}$ be a set of $\alpha$-vectors, each corresponding to a policy with horizon $t-1$ (note that this is not necessarily the optimal set).

A shorter way to write the value function, which often appears in the preliminary section of papers, is \cite{lee2007makes}
\begin{align}
V_t(b) = \max_{a\in A}\left[ \sum_{s\in S} R(s,a)b(s) + \gamma \sum_{o\in\Omega} \Pr(o | b,a) V_{t-1}(\tau(b,a,o))   \right]
\end{align}
where $\tau(b,a,o)$ is the belief state after Bayesian filtering on $b$ given $a,o$, and
\begin{align}
\Pr(o | b,a) = \sum_{s\in S}b(s)\sum_{s'\in S}T(s,a,s')O(s',a,o)
\end{align}




% , which defines a mapping from belief space $B$ to value space $\mathbb{R}$.


\section{Problem Formulation}
Broadly and vaguely, for a given ``task'', we are interested in how a POMDP $P_1$ that ``models'' the task can be expanded or modified into another POMDP $P_2$ that has an ``optimal policy'' that is ``better'' than the optimal policy for $P_1$.  I think there are two concrete problem formulations that express this broad problem, and imply principled planning algorithms. Our hope is that we can solve either one of them, and use the implied planning algorithm to do our task, target search in unknown rich environment, more effectively.

\subsection{Problem 1: Expansion of observation space and observation model}

The motivating intuition is that, for the task of ``finding salt'' in an environment that contains not only salt, a POMDP that contains only the observation model for salt will lead to an optimal policy (despite being optimal) worse than a POMDP that also contains an observation model for, e.g. kitchen, because the observation of kitchen's location can inform the location of salt, and the observation of kitchen is likely less noisy. We are interested in characterizing the necessary conditions of the additional observation space and observation model so that the resulting expanded POMDP is guaranteed to have a better optimal policy.

\begin{definition}(Problem 1)
Let $R:S\rightarrow \mathbb{R}$ be a sparse reward function that represents a task of interest, where $s_{goal}\in S$ is the single goal state, and $s_{fail}\in S$ is the single failure state, and $R_{max}=R(s_{goal})$, $R_{min}=R(s_{fail})$, and $R(s)=0, \forall s \not\in\{s_{goal},s_{fail}\}$. Let $P$ be a POMDP $\langle S, A, \Omega, T, O, R, \gamma  \rangle$. Let $\alpha^*$ be the $\alpha$-vector of the optimal policy at the initial belief $b_0$. Let $\td{P}$ be a POMDP $\langle S, A, \td{\Omega}, T, \td{O}, R, \gamma  \rangle$ where $\td{O}=O\cup\{\td{o}\}$ and $\td{O}(s',a,o)=O(s',a,o)\ \forall o\neq \td{o}$. Let $\td{\alpha}^*$ be the $\alpha$-vector of the optimal policy for $\td{P}$ at initial belief $b_0$. What conditions of $b_0$, $\td{o}$ and $\td{O}(s',a,\td{o})$ need to be satisfied so that
\begin{align}
b_0\cdot \td{\alpha}^* - b_0\cdot \alpha^* > 0
\end{align}
\end{definition}




% Let $P$ be a POMDP $\langle S, A, \Omega, T, O, R, \gamma  \rangle$. Let $b_0$ be the initial belief. The optimal policy with horizon $t$ for $P$ at $b_0$ corresponds to an $\alpha$-vector,
% $\alpha^* = \argmax_{\alpha\in\Gamma_t} (\alpha\cdot b_0)$, where $\Gamma_t$ is the set of all $\alpha$-vectors under this horizon.

% Our intuition is that

% Given an initial belief $b_0$, the optimal POMDP policy with horizon $t$ corresponds to an $\alpha$-vector, %$\alpha^*$, defined as
% % \begin{align}

% % \end{align}

% $\td{\alpha}$


\bibliographystyle{plainnat}
\bibliography{references}
\end{document}
